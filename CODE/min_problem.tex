\documentclass[12pt]{article}

% Packages
\usepackage{amsmath}    % For advanced math typesetting
\usepackage{amssymb}    % For additional math symbols
\usepackage{hyperref}   % For hyperlinks
\usepackage{geometry}   % To adjust margins if needed

% Geometry settings (optional)
\geometry{
    a4paper,
    left=25mm,
    right=25mm,
    top=25mm,
    bottom=25mm,
}

\begin{document}

% Section: Singular Control Problem
\section{Singular Control Problem}
\label{sec:singular-control-problem}

% Subsection: Maximization Problem
\subsection{Maximization Problem}
\label{sec:maximization-problem}

The \textbf{value function} of the maximization problem for $j=0,w,b$ (corresponding to the three portfolios: no option, writer, buyer) is defined as:

\begin{equation}
V^j(t,b,y,s) = \sup_{L,M} \;  \mathbb{E}\left[ \; \mathcal{U}\left( \mathcal{W}^{j}_T \right) \; \bigg| \; B_{t} = b, Y_{t} = y, S_{t} = s \right],
\end{equation}

where $\mathcal{U}: \mathbb{R} \to \mathbb{R}$ is a concave increasing \textbf{utility function}. The \textbf{exponential utility} is what we are looking for:

\begin{equation}
\mathcal{U}(x) := 1 - e^{-\gamma x} \quad \quad \gamma > 0.
\end{equation}

% Subsection: Indifference Pricing
\subsection{Indifference Pricing}
\label{sec:indifference-pricing}

The writer (buyer) option price is defined as the amount of cash to add (subtract) to the bank account, 
such that the maximal expected utility of wealth of the writer (buyer) is the same as he could get with 
the zero-option portfolio.

\begin{itemize}
    \item The \textbf{writer price} is the value $p^w > 0$ such that 
    \begin{equation}
        V^0(t,b,y,s) = V^w(t,b+p^w,y,s),
    \end{equation}
    
    \item The \textbf{buyer price} is the value $p^b > 0$ such that
    \begin{equation}
        V^0(t,b,y,s) = V^b(t,b-p^b,y,s).
    \end{equation}
\end{itemize}

% Add more sections or content as needed...

\end{document}
